\section{Пользовательский интерфейс}

\begin{figure}[h]
    \centering

    \includegraphics[width=\textwidth]{common}
    \caption{Общий вид интерфейса}
    \label{fig:common}
\end{figure}

Набор интерфейсов системы (\ref{fig:common}) состоит из
\begin{itemize}
    \item Интерфейса отображения состояния симуляции в псевдографике
    \item Интерфейса редактирования структуры метро
    \item Интерфейса редактирования хода симуляции
    \item Интерфейса отображения вспомогательных характеристик системы
\end{itemize}

\pagebreak

\begin{figure}[h]
    \centering
    \includegraphics[width=\textwidth]{sections}
    \caption{Интерфейс редактирования линий}
    \label{fig:sections}
\end{figure}

Интерфейс редактирования линий (\ref{fig:sections}) позволяет редактировать структуру метро.
Добавлять/удалять линии, станции, секции

\pagebreak

\begin{figure}[]
    \centering
    \includegraphics[width=\textwidth]{trains}
    \caption{Редактирование поездов}
    \label{fig:trains}
\end{figure}

Интерфейс редактирования поездов (\ref{fig:trains}) позволяет настраивать поезда.
Добавлять/удалять, редактировать скорость.
