\section{Уточнение постановки задачи}

Требуется реализовать работу метро.
В данной модели метро представляет из себя набор линий.
Каждая из линий состоит из станций, каждая станция может содержать несколько платформ.
Тоннели соединяют между собой платформы и однонаправлены.
Также каждой линии принадлежит набор поездов, которые по ней движутся.
Каждый поезд содержит граф пути, указывающий порядок прохождения секций (тоннелей).

Также требуется реализовать возможность изменения структуры метро (добавление/удаление линий, станций, секций и поездов).
Кроме того, нужно сделать графический интерфейс, отображающий положение поездов и структуру метро.

Необходимо, чтобы сама модель не зависела от конкретного графического представления.
Для этого будем использовать промежуточное сериализуемое состояние модели.
